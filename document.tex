\documentclass[b5paper,8 pt,leqno]{book}
\usepackage{polski}  
\usepackage[utf8]{inputenc}  
\usepackage{fancyhdr} %naglowki
\usepackage{amssymb}
%\usepackage{amsfonts}
\usepackage[T1]{fontenc}
\usepackage{amsmath}
\usepackage{geometry}
\usepackage{enumerate}
\newgeometry{tmargin=2cm, bmargin=2cm, lmargin=1.8cm, rmargin=1.8 cm}
\usepackage{titlesec}

\begin{document}
\begin{enumerate}
\small
\setcounter{enumi}{3}
\item
	\begin{enumerate}[a)]
		\item 
		\item 
		\newpage
		
		\item Udowodnić, że 
		\begin{equation*}
			[T(B)]' = T(B').
		\end{equation*}
		\item Z b) i c) wywnioskować, że
		\begin{equation*}
			T(B)\cup T(C) = T(B \cup C).
		\end{equation*}
	\end{enumerate} 
\item Udowodnić, że jeśli algebra jest skończona to:
\begin{enumerate}[a)]
	\item Opisane w twierdzeniu 2 odwzorowywanie przeprowadza różne elementy na różne,
	tzn. jeśli $B\neq C$ to $T(B) \neq T(C)$. \\
	W\ s\ k\ a\ z\ ó\ w\ k\ a\ . Jeżli $B\neq C$, to istnieje atom należący do jednego elementu,
	a nie należący do drugiego.
	
	\item Dla każdego podzbioru S zbioru atomów istnieje element D taki, że $T(D) = S$.\\
	W\ s\ k\ a\ z\ ó\ w\ k\ a\ . Wziąć za $D$ sumę, w sensie $A_1 \cup A_2 \cup \dots \cup A_k$, wszystkich atomów $A_1, \dots, A_k$ należących do $D$.
\end{enumerate}

\item Z twierdzenia 3 wywnioskować, że każda algebra Boole'a skończona ma liczbę elementów postaci $2^n$.

\item Udowodnić, że rachunek zadań w przypadku, gdy liczba zmiennych zdaniowych jest nieskończona, traktowany jako algebra Boole'a tak jak w przykładzie 2, z paragrafu 38, jest algerbą Boole'a, w której żaden element nie ma atomu. Algebry takie nazywamy algebrami \textit{bezatomowymi}.
\end{enumerate}

\normalsize
\textbf{\textsection \, 42. ZNACZENIE TWIERDZEŃ O REPREZENTACJI} \\
\indent Omówmy teraz wnioski jakie płyną z twierdzenia o reprezantacji algebry Boole'a. \\
\indent Algebry Boole'a wprowadziliśmy aksjomatycznie w \textsection31 tego rozdziału, używając języka potocznego. Teorię tę można również przedstawić w postaci elementarnej teorii sformalizowanej (zob. r. IV). \\
\indent Formułami atomowymi są formuły postaci równości dwóch termów.
\begin{equation}
f(x_1, \dots, x_n) = g(x_1, \dots, x_n)
\label{eq1}
\end{equation}
\textit{Termy} są to wyrażenia zbudowane w omówiony sposób ze zmiennych $x_1, x_2, x_3, \dots$ stałych 0 oraz 1 (działań zeroargumentowych), symboli ,,$'$'' działania jednoagrumentowego, oraz symboli ,,$\cup$'' i ,,$\cap$'' działań dwuargumentowych. Na przykład wyrażenia;

\begin{gather*}
\left( (X_1 \cap X_3) \cup (X_2')'\right) \cup (0 \cup X_1),\\
X_1 \cup (X_2 \cup X_3), \quad 1'\cup(0 \cap 0), \quad X_5, \quad X_1', \quad 0, \quad 1 
\end{gather*}

są termami, natomiast wyrażenie:
\begin{equation*}
X_1 \cap \cap '(X_1\cup 0)
\end{equation*}
nie jest termem, gdyż nie jest poprawnie zbudowane.

\indent Z formuł postaci (\ref{eq1}) budujemy inne formuły, łącząc je spójnikami zdaniowymi i opatrując kwantyfikatorami.

\indent Formułą będzie na przykład
\begin{equation*}
\textrm{A}x_1 \; \textrm{E}x_1 \; (x_1 \cup x_2 = x_2) \; \& \sim (0 = x_1),
\end{equation*}
czy też

\begin{equation*}
\textrm{A}x_1 \; (x_1 = x_1).
\end{equation*}

Formuła ta jest zdaniem, gdyż nie zawiera zmiennych wolnych. Lecz już formuła

\begin{equation*}
\textrm{E}x_2[(x_1 \cup x_1 = x_2) \; \& \; (0=x_1)],
\end{equation*}

czy też formuła 
\begin{equation*}
	x_1 = x_1
\end{equation*}
nie są zdaniami, gdyż zawierają zmienną wolną $x_1$.

\indent Aksjomatami teorii będą następujące zdania:
\begin{enumerate}
	\item Aksjomat dotyczący elementów wyróżnionych
	\begin{equation*}
	\thicksim (0 = 1).
	\end{equation*}
	
	\item Aksjomat równości:
	\begin{equation*}
	\textrm{A}x_1 \; (x_1 = x_2), \\
	\textrm{A}x_1\textrm{A}x_2 \; [(x_1 = x_2) \Rightarrow (x_2 = x_1)],
	\textrm{A}x_1\textrm{A}x_2\textrm{A}x_3 \; [(x_1 = x_2) \& (x_2 = x_3) \Rightarrow (x_1 = x_3)],
	\textrm{A}x_1\textrm{A}x_2\textrm{A}x_3 \; [(x_1 = x_2) \Rightarrow {(x_1 \cup x_3 = x_2 \cup x_3) \& \& (x_1 \cap x_3 = x_2 \cap x_3) \& (x_1'=x_2')}].
	\end{equation*}
	
	\item Aksjomat algebry Boole'a:
	
\end{enumerate}
\end{document}